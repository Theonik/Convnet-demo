%\addcontentsline{toc}{chapter}{Development Process}
\chapter{Experiment Implementation}
To implement our experiments we need to make a series of decision.

\section{Choice of Platform}
To start we need to decide how we are going to implement the neural networks. This involved several steps, of course one would be tempted to write their own implementation for an ANN, however that might be a major project on its own so a better solution is required.

We would like to minimise implementation effort in parts that do not improve the final project as much as possible so a library of some sort is necessary.

\subsection{Criteria of a Framework}
For our framework we wanted something that is both performant but also easy to use. We also want support for the latest ANN features while it needs to work on our workstation.

\subsection{Performance}
One of the greatest developments of the past 8 years in ANNs and machine learning in general is the advent and wide availability of parallel computing. Specifically GPU compute solutions like nVidia's CUDA technology have been invaluable in making Deep Neural Network research viable offering orders of magnitude better training times. Our frameword must run on a GPU.

\subsection{Ease of use}
We would like a library that makes testing an idea quickly, two modern candidates emerged at that point Google's new Tensor Flow software was one of them and was our favourite for the early parts of the project. Keras was another. Both use Python to define models and a backend implementation to do processing

\subsection{Features}
We were not looking for specific features in this part of our project, what we were looking for however was a framework that is used in current research and is geared towards experimental CNNs both Keras and Tensor Flow support similar advanced CNN features which makes them fairly equivalent as a choice. They were both alpha software when the project started but so are most libraries in the field as CNNs are fairly new. We initially settled on Tensor Flow.

\subsection{Platform Support}
Most of the frameworks for this task are designed to work on Linux primarily. Tensor Flow works on Linux and OSX. This was a problem when we were running Windows on our main machine so the initial solution was to run Tensor Flow on a Virtual Machine with Linux Mint. This initially appeared to work well but we quickly discovered issues with this approach. To get good training performance CUDA is required. However CUDA requires a direct connection to the graphics adapter.

A solution we considered was to use Intel's VT-d and pass the PCI-E device of the GPU directly to the VM using the integrated Intel GPU in Windows instead allowing us to use it under a VM but this proved to be too complicated to set up and any other solution was equally complex. We had to abandon Tensor Flow. Keras, supported Windows as well so we ended up switching our efforts to that. This was relatively early on in the project's lifecycle so we could afford to make the change.

\section{The Stack}
In our architecture we ended up having several layers of software contributing to our final software stack.

\subsection{GPU Drivers and CUDA}
Because of the task of training and running through a neural network is an inherently parallel task we use Nvidia's CUDA technology in order to run computations on the GPU which produces at least an order of magnitude in performance gains over a modern traditional CPU architecture.

Nvidia's CUDA API has emerged as the dominant solution for scientific and industrial compute applications after competing open APIs like OpenCL failed to gain any traction. If you want good ANN performance CUDA is necessary.

\subsection{CuDNN}
CuDNN is an nVidia library written in C++ which provides performance optimisations for several ANN computations to run on CUDA enabled GPUs. This library isn't necessary, but it provided us with a significant performance boost so ended up being invaluable in our experiments.